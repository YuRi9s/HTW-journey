\documentclass{article}
\usepackage{pgfplots}
\usepackage{enumitem}
\usepackage{fancyhdr}
\usepackage{extramarks}
\usepackage[ngerman]{babel}
\usepackage{amsthm}
\usepackage{amsfonts}
\usepackage[T1]{fontenc}
\usepackage[plain]{algorithm}
\usepackage{algpseudocode}
\usepackage{xcolor,listings}
\usepackage{textcomp}
\usepackage{tabularx}
\usepackage{geometry}
\usepackage{booktabs}
\usepackage{hyperref}
\usepackage{tikz}
\usepackage{venndiagram}
\usepackage{amsmath, amssymb}
\usepackage{graphicx}
\usepackage{marvosym}
\usepackage{listings,lstautogobble}

\lstnewenvironment{javacode}[1][]
{
    \lstset{
        language=Java,
        basicstyle=\small\ttfamily,
        keywordstyle=\color{blue},
        commentstyle=\color{green!40!black},
        stringstyle=\color{red},
        numbers=left,
        numberstyle=\tiny\color{purple},
        stepnumber=1,
        numbersep=10pt, % Adjusted to move line numbers outside the box
        frame=single,
        linewidth=\linewidth,
        breaklines=true,
        breakatwhitespace=false,
        tabsize=4,
        autogobble=true,
        xleftmargin=0pt, % Adjusted to start from the left margin within the box
        #1 % Options passed to lstset
    }
}
{}

\geometry{a4paper, margin=1in}
\usetikzlibrary{er}
\lstset{upquote=true}
\pgfplotsset{compat=1.18}
\usetikzlibrary{positioning,shapes}
\usetikzlibrary{positioning, arrows.meta}
\usetikzlibrary{automata,positioning}

%
% Basic Document Settings
%

\topmargin=-0.45in
\evensidemargin=0in
\oddsidemargin=0in
\textwidth=6.5in
\textheight=9.0in
\headsep=0.25in

\linespread{1.1}

\pagestyle{fancy}
\lhead{\hmwkAuthorName}
\chead{\hmwkClass}
\rhead{\firstxmark}
\lfoot{\lastxmark}
\cfoot{\thepage}

\renewcommand\headrulewidth{0.4pt}
\renewcommand\footrulewidth{0.4pt}

\setlength\parindent{0pt}

%
% Create Review Sections
%

\newcommand{\enterReviewHeader}[1]{
    \nobreak\extramarks{}{Review \arabic{#1} continued on next page\ldots}\nobreak{}
    \nobreak\extramarks{Review \arabic{#1} (continued)}{Review \arabic{#1} continued on next page\ldots}\nobreak{}
}

\newcommand{\exitReviewHeader}[1]{
    \nobreak\extramarks{Review \arabic{#1} (continued)}{Review \arabic{#1} continued on next page\ldots}\nobreak{}
    \stepcounter{#1}
    \nobreak\extramarks{Review \arabic{#1}}{}\nobreak{}
}

\setcounter{secnumdepth}{0}
\newcounter{partCounter}
\newcounter{reviewProblemCounter}
\setcounter{reviewProblemCounter}{1}
\nobreak\extramarks{Review \arabic{reviewProblemCounter}}{}\nobreak{}

%
% Review Problem Environment
%
% This environment takes an optional argument. When given, it will adjust the
% review counter. This is useful for when the problems given for your
% assignment aren't sequential. See the last 3 problems of this template for an
% example.
%
\newenvironment{reviewProblem}[1][-1]{
    \ifnum#1>0
        \setcounter{reviewProblemCounter}{#1}
    \fi
    \section{Review \arabic{reviewProblemCounter}}
    \setcounter{partCounter}{1}
    \enterReviewHeader{reviewProblemCounter}
}{
    \exitReviewHeader{reviewProblemCounter}
}

%
% Homework Details
%   - Title
%   - Due date
%   - Class
%   - Section/Time
%   - Instructor
%   - Author
%

\newcommand{\hmwkTitle}{}
\newcommand{\hmwkDueDate}{jun 5, 2024}
\newcommand{\hmwkClass}{Programmierung 2 (PIB-PRG2/DFIW-PRG2)}
\newcommand{\hmwkClassInstructor}{Prof. Markus Esch}
\newcommand{\hmwkAuthorName}{\textbf{}}

%
% Title Page
%

\title{
    \vspace{3in}
    \textmd{\textbf{\hmwkClass}}\\
    \normalsize\vspace{0.2in}\small{Due\ on\ \hmwkDueDate\ at 20:00}\\
    \vspace{0.2in}\large{\textit{\hmwkClassInstructor}}
    \vspace{3in}
}

\author{\hmwkAuthorName}

\renewcommand{\part}[1]{\textbf{\large Part \Alph{partCounter}}\stepcounter{partCounter}\\}

%
% Various Helper Commands
%

% Useful for algorithms
\newcommand{\alg}[1]{\textsc{\bfseries \footnotesize #1}}

% For derivatives
\newcommand{\deriv}[1]{\frac{\mathrm{d}}{\mathrm{d}x} (#1)}

% For partial derivatives
\newcommand{\pderiv}[2]{\frac{\partial}{\partial #1} (#2)}

% Integral dx
\newcommand{\dx}{\mathrm{d}x}

% Alias for the Solution section header
\newcommand{\solution}{\textbf{\large Solution}}

% Probability commands: Expectation, Variance, Covariance, Bias
\newcommand{\E}{\mathrm{E}}
\newcommand{\Var}{\mathrm{Var}}
\newcommand{\Cov}{\mathrm{Cov}}
\newcommand{\Bias}{\mathrm{Bias}}

\begin{document}

\maketitle

\pagebreak

% Review 1
\begin{reviewProblem}
    \section*{Review-Report zur Übung Nr.5}
    \subsection*{Coding-Team: 46}
    \subsubsection*{Review erhalten von: Team 45}

    \begin{enumerate}
        \item \textbf{Erläutern Sie, welche Aspekte im Review als positiv bewertet wurden:}
              \begin{enumerate}[label=\alph*.]
                  \item Mein Code war schön und lesbar.
                  \item In der Java-Klasse "MyFunction" habe ich den ternären Operator verwendet, um den Code sauberer und besser lesbar zu machen.
                  \item Meine Dokumentation war gut strukturiert.
              \end{enumerate}
    
        \item \textbf{Erläutern Sie, welche Anforderungen als unvollständig oder nicht umgesetzt identifiziert wurden und welche Hinweise Ihnen gegeben wurden, um die Anforderungen umzusetzen:}
              \begin{enumerate}[label=\alph*.]
                  \item Die Aufgabe "d" wurde nicht vollständig erfüllt.
                  \item Ich habe (evenOdd) als Typ (int) verwendet und nicht "Predict".
                  \item Aufgabe "c" wurde nicht verwendet, um Aufgabe ("d")  zulösen.
                  \item In der Lambda-Ausdruck für Fibonacci wurde nicht benutzt.
                  \item In "conditionateOutput" und "conditionateInput" habe ich (static) und nicht "default" verwendet.
              \end{enumerate}
    
        \item \textbf{Welche Hinweise wurden Ihnen gegeben, um den Code sauberer, lesbarer und wartbarer zu gestalten?}
              \begin{enumerate}[label=\alph*.]
                  \item Abgesehen davon, die negativen Aspekte zu korrigieren, war mein Code in Ordnung.
              \end{enumerate}
    
        \item \textbf{Welche weiteren Hinweise wurden gegeben?}
              \begin{enumerate}[label=\alph*.]
                  \item Die Aufgabenstellung besser lesen.
              \end{enumerate}
    
        \item \textbf{Welche Fragen zu Ihrem Code konnten Sie nicht beantworten?}
              \begin{enumerate}[label=\alph*.]
                  \item Alle Fragen wurden beantwortet.
              \end{enumerate}
    
        \item \textbf{Gibt es Aspekte der Übung, die Sie auch nach der Bearbeitung des Übungsblattes und nach dem Code Review nicht verstanden haben? Wenn ja, welche nicht?}
              \begin{enumerate}[label=\alph*.]
                  \item Alles war klar und verständlich.
              \end{enumerate}
    
        \item \textbf{Haben Sie die Lernziele des Übungsblattes erreicht? Wenn nein, welche nicht?}
              \begin{enumerate}[label=\alph*.]
                  \item Ja.
              \end{enumerate}
    \end{enumerate}
    
    \section*{Coding-Team: 46}
    \begin{minipage}[t]{0.45\textwidth}
        Yahya Elshekh Selo\\
        Matr. 5013655
    \end{minipage}
    \hfill
    \begin{minipage}[t]{0.45\textwidth}
        Sherwan Diko\\
        Matr. 5013656
    \end{minipage}
\end{reviewProblem}

\end{document}
