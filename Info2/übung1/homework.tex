\documentclass{article}
\usepackage{pgfplots}
\usepackage{enumitem}
\usepackage[utf8]{inputenc}
\usepackage{fancyhdr}
\usepackage{extramarks}
\usepackage[ngerman]{babel}
\usepackage{amsthm}
\usepackage{amsfonts}
\usepackage[T1]{fontenc}
\usepackage[plain]{algorithm}
\usepackage{algpseudocode}
\usepackage{xcolor,listings}
\usepackage{textcomp}
\usepackage{tabularx}
\usepackage{geometry}
\usepackage{booktabs}
\usepackage{hyperref}
\usepackage{tikz}
\usepackage{venndiagram}
\usepackage{amsmath, amssymb}
\usepackage{graphicx}
\usepackage{marvosym}
\usepackage{listings,lstautogobble}



\lstnewenvironment{javacode}[1][]
{
    \lstset{
        language=Java,
        basicstyle=\small\ttfamily,
        keywordstyle=\color{blue},
        commentstyle=\color{green!40!black},
        stringstyle=\color{red},
        numbers=left,
        numberstyle=\tiny\color{purple},
        stepnumber=1,
        numbersep=10pt, % Adjusted to move line numbers outside the box
        frame=single,
        linewidth=\linewidth,
        breaklines=true,
        breakatwhitespace=false,
        tabsize=4,
        autogobble=true,
        xleftmargin=0pt, % Adjusted to start from the left margin within the box
        #1 % Options passed to lstset
    }
}
{}
\usetikzlibrary{graphs, arrows.meta}
\geometry{a4paper, margin=1in}
\usetikzlibrary{er}
\lstset{upquote=true}
\pgfplotsset{compat=1.18}
\usetikzlibrary{positioning,shapes}
\usetikzlibrary{positioning, arrows.meta}
\usetikzlibrary{automata,positioning}

%
% Basic Document Settings
%

\topmargin=-0.45in
\evensidemargin=0in
\oddsidemargin=0in
\textwidth=6.5in
\textheight=9.0in
\headsep=0.25in

\linespread{1.1}

\pagestyle{fancy}
\lhead{\hmwkAuthorName}
\chead{\hmwkClass}
\rhead{\firstxmark}
\lfoot{\lastxmark}
\cfoot{\thepage}

\renewcommand\headrulewidth{0.4pt}
\renewcommand\footrulewidth{0.4pt}

\setlength\parindent{0pt}

%
% Create Problem Sections
%

\newcommand{\enterReviewHeader}[1]{
    \nobreak\extramarks{}{Problem \arabic{#1} continued on next page\ldots}\nobreak{}
    \nobreak\extramarks{Problem \arabic{#1} (continued)}{Problem \arabic{#1} continued on next page\ldots}\nobreak{}
}

\newcommand{\exitReviewHeader}[1]{
    \nobreak\extramarks{Problem \arabic{#1} (continued)}{Problem \arabic{#1} continued on next page\ldots}\nobreak{}
    \stepcounter{#1}
    \nobreak\extramarks{Problem \arabic{#1}}{}\nobreak{}
}

\setcounter{secnumdepth}{0}
\newcounter{partCounter}
\newcounter{reviewProblemCounter}
\setcounter{reviewProblemCounter}{1}
\nobreak\extramarks{Problem \arabic{reviewProblemCounter}}{}\nobreak{}

%
% Problem Problem Environment
%
% This environment takes an optional argument. When given, it will adjust the
% review counter. This is useful for when the Problems given for your
% assignment aren't sequential. See the last 3 Problems of this template for an
% example.
%
\newenvironment{reviewProblem}[1][-1]{
    \ifnum#1>0
        \setcounter{reviewProblemCounter}{#1}
    \fi
    \section{Problem \arabic{reviewProblemCounter}}
    \setcounter{partCounter}{1}
    \enterReviewHeader{reviewProblemCounter}
}{
    \exitReviewHeader{reviewProblemCounter}
}

%
% Homework Details
%   - Title
%   - Due date
%   - Class
%   - Section/Time
%   - Instructor
%   - Author
%

\newcommand{\hmwkTitle}{}
\newcommand{\hmwkDueDate}{May 22, 2024}
\newcommand{\hmwkClass}{Informatik 2 (PIB-INF2)}
\newcommand{\hmwkClassInstructor}{Prof. Klaus Berberich}
\newcommand{\hmwkAuthorName}{\textbf{Yahya Selo}}

%
% Title Page
%

\title{
    \vspace{3in}
    \textmd{\textbf{\hmwkClass}}\\
    \normalsize\vspace{0.2in}\small{Due\ on\ \hmwkDueDate\ at 10:00}\\
    \vspace{0.2in}\large{\textit{\hmwkClassInstructor}}
    \vspace{3in}
}

\author{\hmwkAuthorName}

\renewcommand{\part}[1]{\textbf{\large Part \Alph{partCounter}}\stepcounter{partCounter}\\}

%
% Various Helper Commands
%

% Useful for algorithms
\newcommand{\alg}[1]{\textsc{\bfseries \footnotesize #1}}

% For derivatives
\newcommand{\deriv}[1]{\frac{\mathrm{d}}{\mathrm{d}x} (#1)}

% For partial derivatives
\newcommand{\pderiv}[2]{\frac{\partial}{\partial #1} (#2)}

% Integral dx
\newcommand{\dx}{\mathrm{d}x}

% Alias for the Solution section header
\newcommand{\solution}{\textbf{\large Solution}}

% Probability commands: Expectation, Variance, Covariance, Bias
\newcommand{\E}{\mathrm{E}}
\newcommand{\Var}{\mathrm{Var}}
\newcommand{\Cov}{\mathrm{Cov}}
\newcommand{\Bias}{\mathrm{Bias}}

\begin{document}

\maketitle
\pagebreak
\begin{reviewProblem}
    \subsection*{Hash-Tabellen}
    Hash-Tabelle mit Verkettung stehen 10 Speicherplätze zur Verfügung
    \section*{Question 1(a)}
    Zuerst berechnen wir für jeden Schlüsselwert den Hash-Wert und ordnen ihn dann dem entsprechenden Speicherplatz zu.
    
    Die Hash-Funktion lautet:
    \[
        h(k) = (k \mod 43) \mod 10
    \]
    
    \begin{enumerate}
        \item Für \( k = 100 \):
              \[
                  100 \mod 43 = 14 \quad \text{und} \quad 14 \mod 10 = 4
              \]
              Also wird der Schlüsselwert 100 im Speicherplatz 4 abgelegt.
              
        \item Für \( k = 200 \):
              \[
                  200 \mod 43 = 28 \quad \text{und} \quad 28 \mod 10 = 8
              \]
              Also wird der Schlüsselwert 200 im Speicherplatz 8 abgelegt.
              
        \item Für \( k = 300 \):
              \[
                  300 \mod 43 = 42 \quad \text{und} \quad 42 \mod 10 = 2
              \]
              Also wird der Schlüsselwert 300 im Speicherplatz 2 abgelegt.
              
        \item Für \( k = 400 \):
              \[
                  400 \mod 43 = 11 \quad \text{und} \quad 11 \mod 10 = 1
              \]
              Also wird der Schlüsselwert 400 im Speicherplatz 1 abgelegt.
              
        \item Für \( k = 500 \):
              \[
                  500 \mod 43 = 25 \quad \text{und} \quad 25 \mod 10 = 5
              \]
              Also wird der Schlüsselwert 500 im Speicherplatz 5 abgelegt.
              
        \item Für \( k = 600 \):
              \[
                  600 \mod 43 = 39 \quad \text{und} \quad 39 \mod 10 = 9
              \]
              Also wird der Schlüsselwert 600 im Speicherplatz 9 abgelegt.
              
        \item Für \( k = 700 \):
              \[
                  700 \mod 43 = 7 \quad \text{und} \quad 7 \mod 10 = 7
              \]
              Also wird der Schlüsselwert 700 im Speicherplatz 7 abgelegt.
              
        \item Für \( k = 800 \):
              \[
                  800 \mod 43 = 21 \quad \text{und} \quad 21 \mod 10 = 1
              \]
              Also wird der Schlüsselwert 800 im Speicherplatz 1 abgelegt. Da der Speicherplatz 1 bereits von 400 belegt ist, wird 800 verkettet.
              
        \item Für \( k = 900 \):
              \[
                  900 \mod 43 = 35 \quad \text{und} \quad 35 \mod 10 = 5
              \]
              Also wird der Schlüsselwert 900 im Speicherplatz 5 abgelegt. Da der Speicherplatz 5 bereits von 500 belegt ist, wird 900 verkettet.
              
        \item Für \( k = 1000 \):
              \[
                  1000 \mod 43 = 9 \quad \text{und} \quad 9 \mod 10 = 9
              \]
              Also wird der Schlüsselwert 1000 im Speicherplatz 9 abgelegt. Da der Speicherplatz 9 bereits von 600 belegt ist, wird 1000 verkettet.
    \end{enumerate}
    
    \subsection*{Zusammenfassung der Belegung der Speicherplätze:}
    
    \begin{table}[htbp]
        \centering
        \begin{tabular}{|c|c|}
            \hline
            \textbf{Speicherplatz} & \textbf{Schlüsselwerte} \\ \hline
            0                      & leer                    \\ \hline
            1                      & 400, 800                \\ \hline
            2                      & 300                     \\ \hline
            3                      & leer                    \\ \hline
            4                      & 100                     \\ \hline
            5                      & 500, 900                \\ \hline
            6                      & leer                    \\ \hline
            7                      & 700                     \\ \hline
            8                      & 200                     \\ \hline
            9                      & 600, 1000               \\ \hline
        \end{tabular}
        \caption{Belegung der Speicherplätze}
        \label{table:hash_table}
    \end{table}
    
    \section*{Question 1(b)}
    
    Bei der Verwendung von Hash-Tabellen kann es vorkommen, dass manche Speicherplätze mehrfach und andere gar nicht belegt sind. Wählen Sie 10 Schlüsselwerte, sodass diese beim Einfügen in die Hash-Tabelle alle auf denselben Speicherplatz gehasht werden. Verwenden Sie die Hash-Tabelle aus Aufgabenteil (a).
    
    \subsection*{Lösung}
    
    Wir wollen 10 Schlüsselwerte finden, die alle auf denselben Speicherplatz gehasht werden. Betrachten wir Speicherplatz 0 und wählen Schlüsselwerte \( k \), sodass \( (k \mod 43) \mod 10 = 0 \).
    
    Es muss also gelten \( k \mod 43 = 0, 10, 20, 30 \). Um sicherzustellen, dass sie beim Einfügen alle auf denselben Speicherplatz gehasht werden, können wir \( k \) als:
    
    \[
        k = 43 \times n
    \]
    
    \[
        k \mod 43 = 0 \quad \text{und} \quad 0 \mod 10 = 0
    \]
    
    Hier sind 10 Schlüsselwerte, die alle auf Speicherplatz 0 gehasht werden:
    
    \[
        k = 0, 43, 86, 129, 172, 215, 258, 301, 344, 387
    \]
    
    Für alle diese Werte gilt:
    
    \[
        k \mod 43 = 0 \quad \text{und} \quad 0 \mod 10 = 0
    \]
    
    \begin{table}[htbp]
        \centering
        \begin{tabular}{|c|c|}
            \hline
            \textbf{Speicherplatz} & \textbf{Schlüsselwerte}                      \\ \hline
            0                      & 0, 43, 86, 129, 172, 215, 258, 301, 344, 387 \\ \hline
            1                      & leer                                         \\ \hline
            2                      & leer                                         \\ \hline
            3                      & leer                                         \\ \hline
            4                      & leer                                         \\ \hline
            5                      & leer                                         \\ \hline
            6                      & leer                                         \\ \hline
            7                      & leer                                         \\ \hline
            8                      & leer                                         \\ \hline
            9                      & leer                                         \\ \hline
        \end{tabular}
        \caption{Belegung der Speicherplätze für Teil (b)}
        \label{table:hash_table_b}
    \end{table}
    
    
\end{reviewProblem}
\pagebreak

\begin{reviewProblem}
    \subsection*{Grundbegriffe Graphen}
    \begin{tikzpicture}[>=Stealth, node distance=3cm, on grid, auto]
        % Define nodes
        \node[state] (0) {0};
        \node[state] (1) [above right=of 0] {1};
        \node[state] (2) [right=of 0] {2};
        \node[state] (3) [right=of 1] {3};
        \node[state] (5) [right=of 2] {5};
        \node[state] (4) [below right=of 0] {4};
        
        % Define edges (example edges, you can customize)
        \path[->] (0) edge (1)
        (0) edge (2)
        (1) edge (3)
        (2) edge (1)
        (3) edge (2)
        (3) edge[bend left] (4)
        (4) edge (0)
        (5) edge (3);
    \end{tikzpicture}
    \begin{enumerate}[label=\alph*.]
        \item
        \item
        \item
        \item
        \item
        \item
        \item
    \end{enumerate}
\end{reviewProblem}

\pagebreak
\begin{reviewProblem}
    \subsection*{Knotengrade bestimmen}
    \begin{enumerate}[label=\alph*.]
        \item
        \item
    \end{enumerate}
\end{reviewProblem}
\pagebreak

\pagebreak
\begin{reviewProblem}
    \subsection*{Vollständiger Graph}
    
\end{reviewProblem}
\pagebreak

\begin{reviewProblem}
    \subsection*{Knoten zählen}
    \begin{enumerate}[label=\alph*.]
        \item
        \item
        \item
    \end{enumerate}
\end{reviewProblem}

\end{document}
