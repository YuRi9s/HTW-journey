\documentclass{article}
\usepackage{pgfplots}
\usepackage{enumitem}
\usepackage{fancyhdr}
\usepackage{extramarks}
\usepackage[english]{babel}
\usepackage{amsthm}
\usepackage{amsfonts}
\usepackage[T1]{fontenc}
\usepackage[plain]{algorithm}
\usepackage{algpseudocode}
\usepackage{xcolor,listings}
\usepackage{textcomp}
\usepackage{tabularx}
\usepackage{geometry}
\usepackage{booktabs}
\usepackage{hyperref}
\usepackage{tikz}
\usepackage{venndiagram}
\usepackage{amsmath, amssymb}
\usepackage{graphicx}
\usepackage{marvosym}
\usepackage{listings,lstautogobble}

\lstnewenvironment{javacode}[1][]
{
    \lstset{
        language=Java,
        basicstyle=\small\ttfamily,
        keywordstyle=\color{blue},
        commentstyle=\color{green!40!black},
        stringstyle=\color{red},
        numbers=left,
        numberstyle=\tiny\color{purple},
        stepnumber=1,
        numbersep=10pt, % Adjusted to move line numbers outside the box
        frame=single,
        linewidth=\linewidth,
        breaklines=true,
        breakatwhitespace=false,
        tabsize=4,
        autogobble=true,
        xleftmargin=0pt, % Adjusted to start from the left margin within the box
        #1 % Options passed to lstset
    }
}
{}

\geometry{a4paper, margin=1in}
\usetikzlibrary{er}
\lstset{upquote=true}
\pgfplotsset{compat=1.18}
\usetikzlibrary{positioning,shapes}
\usetikzlibrary{positioning, arrows.meta}
\usetikzlibrary{automata,positioning}


%
% Basic Document Settings
%

\topmargin=-0.45in
\evensidemargin=0in
\oddsidemargin=0in
\textwidth=6.5in
\textheight=9.0in
\headsep=0.25in

\linespread{1.1}

\pagestyle{fancy}
\lhead{\hmwkAuthorName}
\chead{\hmwkClass}
\rhead{\firstxmark}
\lfoot{\lastxmark}
\cfoot{\thepage}

\renewcommand\headrulewidth{0.4pt}
\renewcommand\footrulewidth{0.4pt}

\setlength\parindent{0pt}

%
% Create Review Sections
%

\newcommand{\enterReviewHeader}[1]{
    \nobreak\extramarks{}{Review \arabic{#1} continued on next page\ldots}\nobreak{}
    \nobreak\extramarks{Review \arabic{#1} (continued)}{Review \arabic{#1} continued on next page\ldots}\nobreak{}
}

\newcommand{\exitReviewHeader}[1]{
    \nobreak\extramarks{Review \arabic{#1} (continued)}{Review \arabic{#1} continued on next page\ldots}\nobreak{}
    \stepcounter{#1}
    \nobreak\extramarks{Review \arabic{#1}}{}\nobreak{}
}

\setcounter{secnumdepth}{0}
\newcounter{partCounter}
\newcounter{reviewProblemCounter}
\setcounter{reviewProblemCounter}{1}
\nobreak\extramarks{Review \arabic{reviewProblemCounter}}{}\nobreak{}

%
% Review Problem Environment
%
% This environment takes an optional argument. When given, it will adjust the
% review counter. This is useful for when the problems given for your
% assignment aren't sequential. See the last 3 problems of this template for an
% example.
%
\newenvironment{reviewProblem}[1][-1]{
    \ifnum#1>0
        \setcounter{reviewProblemCounter}{#1}
    \fi
    \section{Review \arabic{reviewProblemCounter}}
    \setcounter{partCounter}{1}
    \enterReviewHeader{reviewProblemCounter}
}{
    \exitReviewHeader{reviewProblemCounter}
}

%
% Homework Details
%   - Title
%   - Due date
%   - Class
%   - Section/Time
%   - Instructor
%   - Author
%

\newcommand{\hmwkTitle}{}
\newcommand{\hmwkDueDate}{unkown}
\newcommand{\hmwkClass}{Industrial Ecology (PIB-INEC/KIB-INEC/MAB.4.2.6.4)}
\newcommand{\hmwkClassInstructor}{Steven P. Frysinger }
\newcommand{\hmwkAuthorName}{\textbf{Yahya Selo}}

%
% Title Page
%

\title{
    \vspace{3in}
    \textmd{\textbf{\hmwkClass}}\\
    \normalsize\vspace{0.2in}\small{Due\ on\ \hmwkDueDate\ at 20:00}\\
    \vspace{0.2in}\large{\textit{\hmwkClassInstructor}}
    \vspace{3in}
}

\author{\hmwkAuthorName}

\renewcommand{\part}[1]{\textbf{\large Part \Alph{partCounter}}\stepcounter{partCounter}\\}

%
% Various Helper Commands
%

% Useful for algorithms
\newcommand{\alg}[1]{\textsc{\bfseries \footnotesize #1}}

% For derivatives
\newcommand{\deriv}[1]{\frac{\mathrm{d}}{\mathrm{d}x} (#1)}

% For partial derivatives
\newcommand{\pderiv}[2]{\frac{\partial}{\partial #1} (#2)}

% Integral dx
\newcommand{\dx}{\mathrm{d}x}

% Alias for the Solution section header
\newcommand{\solution}{\textbf{\large Solution}}

% Probability commands: Expectation, Variance, Covariance, Bias
\newcommand{\E}{\mathrm{E}}
\newcommand{\Var}{\mathrm{Var}}
\newcommand{\Cov}{\mathrm{Cov}}
\newcommand{\Bias}{\mathrm{Bias}}

\begin{document}


\maketitle


\pagebreak
\tableofcontents

\section*{Team}
\begin{minipage}[t]{0.3\textwidth}
    Yahya Elshekh Selo\\
    Matr. 5013655
\end{minipage}
\hfill
\begin{minipage}[t]{0.3\textwidth}
    Sherwan Diko\\
    Matr. 5013656
\end{minipage}
\hfill
\begin{minipage}[t]{0.3\textwidth}
    Mohamad Ata Suleiman\\
    Matr. 3855619
\end{minipage}

\pagebreak
\begin{reviewProblem}

    \section{Introduction}
    \begin{itemize}
        \item Definition of E-Waste
        \item Importance of E-Waste Recycling
        \item Overview of the global e-waste problem
    \end{itemize}

    \section{Sources of E-Waste}
    \begin{itemize}
        \item Common types of e-waste (e.g., computers, smartphones, televisions, etc.)
        \item Major contributors to e-waste (households, businesses, government agencies)
        \item Trends in e-waste generation
    \end{itemize}

    \section{Environmental and Health Impacts}
    \begin{itemize}
        \item Toxic substances in e-waste (e.g., lead, mercury, cadmium, etc.)
        \item Environmental impact of improper disposal (soil, water, air pollution)
        \item Health risks associated with e-waste exposure (both in recycling facilities and through environmental contamination)
    \end{itemize}

    \section{Current E-Waste Recycling Practices}
    \begin{itemize}
        \item Collection methods (e.g., curbside pickup, drop-off centers, manufacturer take-back programs)
        \item Recycling processes (manual disassembly, shredding, separation of materials)
        \item Technologies used in e-waste recycling (mechanical processes, chemical treatments)
    \end{itemize}

    \section{Challenges in E-Waste Recycling}
    \begin{itemize}
        \item Informal recycling sectors and associated risks
        \item Lack of infrastructure and facilities in many regions
        \item Economic challenges (cost of recycling vs. profit from recovered materials)
        \item Regulatory and enforcement issues
    \end{itemize}

    \section{Case Studies}
    \begin{itemize}
        \item Successful e-waste recycling programs (e.g., Europe’s WEEE Directive, Japan’s Home Appliance Recycling Law)
        \item Examples of companies with robust recycling initiatives (e.g., Apple’s recycling robots, Dell’s recycling program)
    \end{itemize}

    \section{Innovative Solutions and Technologies}
    \begin{itemize}
        \item Advances in recycling technology (e.g., robotics, AI sorting, chemical recycling methods)
        \item Design for recycling (products designed to be easily recyclable)
        \item Circular economy approaches (remanufacturing, refurbishment)
    \end{itemize}

    \section{Regulations and Policies}
    \begin{itemize}
        \item Overview of international and national regulations governing e-waste (e.g., Basel Convention, EU directives, national laws)
        \item Extended Producer Responsibility (EPR) policies
        \item Role of government and policy in promoting and enforcing e-waste recycling
    \end{itemize}

    \section{Economic and Social Aspects}
    \begin{itemize}
        \item Economic benefits of e-waste recycling (job creation, recovery of valuable materials)
        \item Social implications (impact on communities, ethical recycling practices)
        \item Collaboration between stakeholders (governments, NGOs, private sector)
    \end{itemize}

    \section{Public Awareness and Education}
    \begin{itemize}
        \item Importance of consumer awareness and participation
        \item Educational campaigns and initiatives to promote recycling
        \item How consumers can responsibly dispose of e-waste
    \end{itemize}

    \section{Future Directions and Recommendations}
    \begin{itemize}
        \item Potential for improving e-waste recycling rates
        \item Recommendations for policy makers, industry, and consumers
        \item Future trends in e-waste generation and management
    \end{itemize}

    \section{Conclusion}
    \begin{itemize}
        \item Summary of key points
        \item The critical role of e-waste recycling in sustainable development
        \item Call to action for increased efforts in e-waste recycling
    \end{itemize}

\end{reviewProblem}

\end{document}
